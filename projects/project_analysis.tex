% Include the header of the document.
\documentclass[a4paper]{article}

% TeX Packages.
\usepackage[T1]{fontenc}              % Use 8-bit T1 fonts
\usepackage[backend=bibtex]{biblatex} % Citing
\usepackage[english]{babel}           % Set English as main language
\usepackage[intoc, english]{nomencl}  % Nomenclature
\usepackage[utf8]{inputenc}           % Allow utf-8 input
\usepackage{algorithmicx}
\usepackage{algorithm}
\usepackage{algpseudocode}
\usepackage{amsfonts}
\usepackage{amsfonts}                 % Blackboard math symbols
\usepackage{amsmath}                  % AMS Math
\usepackage{amssymb}                  % AMS Symbols
\usepackage{amsxtra}
\usepackage{appendix}                 % Appendix
\usepackage{array,epsfig}
\usepackage{booktabs}                 % Professional-quality tables
\usepackage{caption}                  % Captions
\usepackage{color}
\usepackage{csquotes}                 % Context sensitive quotation facilities
\usepackage{float}                    % Float control
\usepackage[top=20mm,
  bottom=20mm,
  left=20mm,
  right=20mm]{geometry}                 % Easily define margins
\usepackage{graphicx}                 % Graphic materials (e.g., images)
\usepackage{hyperref}                 % Hyperlinks
\usepackage{listings}
\usepackage{microtype}                % Microtypography
\usepackage{ntheorem}
\usepackage{rotating}                 % Allow page rotation (e.g., for large table)
\usepackage{subcaption}               % Subcaptions and subfigures.
\usepackage{tikz}                     % Drawings
\usepackage{url}                      % Simple URL typesetting

\date{\today}
\author{Joeri Hermans \\ \href{mailto:joeri.hermans@doct.ulg.ac.be}{joeri.hermans@doct.ulg.ac.be}}

\begin{document}


\title{Sloan Digital Sky Survey Analysis and Data Engineering}
\maketitle

\section{Introduction}
\label{sec:introduction}

The Sloan Digital Sky Survey\footnote{\href{https://www.sdss.org}{https://www.sdss.org}} or SDSS, is a major multi-spectral imaging and spectroscopic redshift survey using a dedicated 2.5 meter wide-angle optical telescope at Apache Point Observatory in New Mexico, United States. The Sloan Digital Sky Survey has created the most detailed three-dimensional maps of the Universe ever made, with deep multi-color images of one third of the sky, and spectra for more than three million astronomical objects. The SDSS began regular survey operations in 2000, after a decade of design and construction.  It has progressed through several phases, SDSS-I (2000-2005), SDSS-II (2005-2008), SDSS-III (2008-2014), and SDSS-IV (2014-).  Each of these phases has involved multiple surveys with interlocking science goals.  The three surveys that comprise SDSS-IV are eBOSS, APOGEE-2, and MaNGA.\\

In this project, you will be working with data from the eBOSS (Extended Baryon Oscillation Spectroscopic Survey) experiment. eBOSS precisely measures the expansion history of the Universe throughout eighty percent of cosmic history, back to when the Universe was less than three billion years old, and improve constraints on the nature of dark energy. “Dark energy” refers to the observed phenomenon that the expansion of the Universe is currently accelerating, which is one of the most mysterious experimental results in modern physics.

\section{Bonus}
\label{sec:bonus}

\emph{TO BE CONFIRMED BY GILLES}

A bonus point (1/20) is awarded to the group which is able to provide the fastest average random query time over the complete eBOSS dataset (171 GB). The query that will be ran to evaluate the performance of your data architecture, will be concerned with selecting objects in a particular region in the sky for a particular redshift. Furthermore, we also might be interested in only selecting particular objects, i.e., \emph{starts}, \emph{galaxies}, and \emph{quasars}.

% Include the footer of the document.
\end{document}

